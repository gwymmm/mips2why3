%\label{ch:test}
We start our verification with an annotated assembly program. This program is 
acceptable as input for the MARS MIPS simulator \footnote{\verb"courses.missouristate.edu/KenVollmar/mars"}. 
Here is an excerpt from an annotated example program: 
\begin{lstlisting}
main:
addi $sp, $sp, -4
sw $ra, 0 ($sp)
addi $t0, $zero, 10
addi $t1, $zero, 5
#@ call
#> simple_addition t0 t1 v1 ra;
jal simple_addition
#> assert {"expl: v1 in [0, 20]" (!v1 >= 0) /\ (!v1 <= 20)};
\end{lstlisting}
Furthermore the assembly program can be converted
to a WhyML module with the help of a simple translation program called \verb"mips2why3".
The last code snippet is translated to:
\begin{lstlisting}
sp := addi (!sp) (-4);
mem := sw (!mem) (!ra) (0) (!sp);
t0 := addi (0) (10);
t1 := addi (0) (5);
touch (ra);
simple_addition t0 t1 v1 ra;
assert {"expl: v1 in [0, 20]" (!v1 >= 0) /\ (!v1 <= 20)};
\end{lstlisting}
The definition of the MIPS instructions is in a separate Why3 module and is
imported at the beginning of the program.
\begin{lstlisting}
#> use import mips.MIPS
\end{lstlisting}
Since the program is a WhyML module now, it can be verified with Why3.
The translation program (\verb"mips2why3") and the MIPS instruction model were
written as part of this work. They are available at the GitHub repository.

\section{MIPS Instruction Set Model}

The MIPS instruction set model is a Why3 module. Apart from the definition of 
constants and helper functions it contains abstract WhyML functions that provide 
the specification of the MIPS instructions. 
This, for example, is the definition of the \verb"add" instruction:
\begin{lstlisting}
val add (x: int) (y: int): int
 requires {"expl: first source register in bounds" 
  in_32bit_bounds x }
 requires {"expl: second source register in bounds"
  in_32bit_bounds y }
 requires {"expl: overflow check" 
  let sx = (unsigned_to_signed x) in
  let sy = (unsigned_to_signed y) in
  let r = sx + sy in
  in_32bit_twos_complement_bounds r }
 ensures { result = (mod (x + y) c_2_POW_32) }
\end{lstlisting}
WhyML functions with a function body start with the \verb"let" keyword (see p. \pageref{function}).
Abstract functions, on the contrary, do not have a body and begin with the 
\verb"val" keyword. Specification constructs (assertions, preconditions, postconditions,
loop invariants and variants) can contain strings that describe them. 
\begin{lstlisting}
"expl: first source register in bounds" 
\end{lstlisting}
When such a string begins with "expl:", its content appears in an interactive Why3 IDE session
instead of the default name for an assertion. \\
Furthermore let bindings can be used in Why3 to decompose an expression into smaller parts. 
\begin{lstlisting}
let sx = (unsigned_to_signed x) in
let sy = (unsigned_to_signed y) in
let r = sx + sy in
 in_32bit_twos_complement_bounds r
\end{lstlisting}
Let bindings are similar to local constant declarations in imperative languages. \\
Here is an example that shows how a MIPS \verb"add" instruction in MARS syntax is mapped to a WhyML function
invocation:
\begin{lstlisting}
add $t2, $t0, $t1  # MARS syntax

t2 := add (!t0) (!t1);  # WhyML translation
\end{lstlisting}
As we can see in the example above, a MIPS register becomes a reference in WhyML (see p. \pageref{ref}). \\
The MIPS theory module contains the definition of the following MIPS instructions. 
They were implemented according to the description in the MIPS ISA Reference \cite{mipsisa}.\\
\vspace{0.2cm}
\begin{tabular} {|l||l|}
\hline
Addition &  \verb"add, addi, addu, addiu" \\
Subtraction &  \verb"sub, subu" \\
Multiplication & \verb"mult, multu" \\
Division & \verb"divu" \\
Helper Instructions & \verb"mfhi, mflo, lui" \\
Bit Shifts & \verb"srl, sll" \\
Comparison Operations & \verb"slt, slti, sltiu, sltu" \\
Data Transfer &  \verb"lw, sw, lb, lbu, sb" \\ 
Bitwise Logical Instructions & \verb"and, andi, or, ori" \\ 
Conditional Branches & \verb"beq, bne" \\
\hline
\end{tabular} \\
The \verb"j", \verb"jal" and \verb"jr" instructions are not modeled explicitly but implicitly 
through the control flow of WhyML. 
The next parts describe the general concepts of the MIPS theory. 
%\pageref{ch:test}
More information can be found on GitHub.

\subsection{Euclidean Modulo Arithmetic}
To describe arithmetic instructions we need the Euclidean modulo operation.
Modulo arithmetic describes integer division. The core parts of this theory are
the two functions $\mathrm{div}(a,n)$ and $\mathrm{mod}(a, n)$. The function $\mathrm{div}(a,n)$ is the result
of an integer division and $\mathrm{mod}(a, n)$ is the remainder. Essentially these functions
are defined by the following equation.
\[ a = n * \mathrm{div}(a, n) + \mathrm{mod}(a, n)\]
There exist numerous conventions that define the functions mod and div. These conventions
define different rules for rounding, resulting in different function results in certain cases.
Nevertheless all conventions lead to the same results when $a$ and $n$ are positive \cite{eucmod}.
In this work only the Euclidean convention will be used. Why3 has a built-in theory 
for Euclidean division.
\begin{lstlisting}
use import int.EuclideanDivision
\end{lstlisting}
There it defines the div and mod functions.
\begin{lstlisting}
function div int int : int
function mod int int : int
\end{lstlisting}
The documentation of the theory says that the result of the mod function must always be non-negative. 
The result of the division is rounded down to the nearest integer when the divisor $n$ is positive.
The result is rounded up when $n$ is negative. In this work $n$ will always be positive. 
We can try to compute $\mathrm{mod}(7, 3)$ and $\mathrm{div}(7,3)$ according to this definition. Computing $7 / 3$ leads to
$2.\overline{3}$. Since $n$ is positive we round down to the nearest integer and get $\mathrm{div}(7, 3) = 2$. To compute the 
remainder we substitute $a$, $n$ and $\mathrm{div}(a, n)$ in the above equation. We get $7 = 3 * 2 + \mathrm{mod}(7, 3)$ 
and solving the equation for $\mathrm{mod}(7, 3)$ leads to $\mathrm{mod}(7, 3) = 1$. \\
Now we can do the same for $-7$. Thus $-7 / 3$ is $-2.\overline{3}$ and rounding leads to $\mathrm{div}(-7,3) = -3$. And
$-7 = 3 * (-3) + \mathrm{mod}(-7, 3)$ leads to $\mathrm{mod}(-7, 3) = 2$.\\
The built-in theory of Why3 leads to the same results.
\begin{lstlisting}
assert { (div 7 3) = 2 };
assert { (mod 7 3) = 1 };
assert { (div (-7) 3) = -3 };
assert { (mod (-7) 3) = 2};
\end{lstlisting}
We need this theory because the arithmetic logical unit (ALU) of a processor
performs Euclidean modulo arithmetic. Essentially a calculation like $ 0 - 1 $ means
$ \mathrm{mod}( (0 - 1), 2^{n} )$ where $n$ is the number of bits in a register.
So we get the characteristic $ 0 - 1 = 2^{n}-1 $ as result of our computation.
The example below shows how the mod function behaves for $n=3$.\vspace{0.2cm}

\begin{tabular} {|c||c|c|c|c|c|c|c|c|c|c|c|c|c|c|c|}
 \hline
 $i$ & -7 & -6 & -5 & -4 & -3 & -2 & -1 & 0 & 1 & 2 & 3 & 4 & 5 & 6 & 7 \\
 $\mathrm{mod}(i, 8)$ & 1 & 2 & 3 & 4 & 5 & 6 & 7 & 0 & 1 & 2 & 3 & 4 & 5 & 6 & 7 \\
 \hline
\end{tabular} \\

\subsection{Addition and Subtraction}

All MIPS instructions for addition and subtraction, whether unsigned or not, 
perform essentially the same operation, an addition or subtraction modulo $2^{32}$. 
The only difference is, that in the case of a signed operation the result is interpreted differently. 
In our MIPS theory a register (of type
\verb"ref int") holds a 32-bit unsigned number. So the value of a register is always in the range $[0, 2^{32}-1]$. 
Each instruction has an appropriate precondition to ensure that this assumption holds. 
When we want to interpret the
content of a register as a signed value we can use the \verb"unsigned_to_signed" conversion function.
\begin{lstlisting}
t0 := 0xffff_ffff;
assert { (unsigned_to_signed !t0) = -1 }; 
\end{lstlisting}  
An overflow check is introduced where needed. To check for overflow the calculation is performed with unbounded integers to
see whether the result fits in the bounds of a 32-bit register.

\subsection{Multiplication and Division}

The instructions for multiplication and division store their results into the \verb"hi" an \verb"lo" registers. The \verb"multu" 
instruction for example performs a multiplication and stores the most significant 32 bits of the result in \verb"hi"
and the least significant 32 bits in \verb"lo". No overflow can occur since 64 bits are enough to store the product of
 two 32-bit numbers. Division stores the quotient into the \verb"lo" and the remainder into the \verb"hi" register. 
 A check is introduced that no division by zero occurs.\\
Note that the registers \verb"hi" and \verb"lo" are passed as references and not by value like most other instruction operands.
\begin{lstlisting}
val multu (x: int) (y: int) (lo: ref int) (hi: ref int): unit
\end{lstlisting}
Since the function is abstract, Why3 is not able to analyze the function body to determine whether \verb"lo" and \verb"hi" are
changed during evaluation. To provide this information we have to specify the data flow for the function manually.
\begin{lstlisting}
writes {lo, hi}
\end{lstlisting}

\subsection{Bit Shifts}

Bit shifts correspond to a multiplication or division of the register value by a power of 2.
A shift right (\verb"srl") is equivalent to a division by $2^{n}$ where $n$ is the number of bits shifted.
A shift left (\verb"sll") is equivalent to a multiplication by $2^{n}$. Arithmetic overflow does not occur.
The upper bits, in the case of a left shift, or the lower bits, in the case of a right shift, are
discarded instead.

\subsection{Comparison Operations}

The comparison operations compare two values and set a register value according to the result. The destination
register is set to 1 if the comparison evaluated to \verb"True". It is set to 0 if the comparison evaluated to \verb"False".
So the destination register is a "boolean variable". The concept of the comparison instructions is simple.
Nevertheless there are subtle differences concerning signed and unsigned operations. Furthermore sign extension
of immediate values has to be considered.   

\subsection{Data Transfer}

For a model of the data transfer instructions we need a data structure that is suitable to describe the 
random access memory. In Why3 we can use the \verb"map" \verb"int" \verb"int" instantiation of the generic map type from the standard
library.
\begin{lstlisting}
use import map.Map
\end{lstlisting}
The type \verb"map" \verb"int" \verb"int" can be described as an unbound array of integers that is indexed by integers.
After the instantiation we have these two functions to work with a map:
\begin{lstlisting}
function get (m: (map int int)) (i: int) : int
function set (m: (map int int)) (i: int) 
             (v: int) : (map int int)
\end{lstlisting}
The get function returns the value in the map m stored at position i. The set function replaces the value
at position i of a map m with a new value v. Doing this, the set function does not change the original map but
returns a new map instead. The map theory of Why3 describes this behavior which is typical for arrays in 
programming. Consequently we can perform write and read accesses like in the following example.
\begin{lstlisting}
let f (m: ref (map int int) ) : unit
=
 m := set !m 5 33;
 assert { (get !m 5) = 33 };
 ()
\end{lstlisting}
This is also the concept behind the Why3 model for the data transfer instructions. Note that the stored elements
in the map are word values. Bytes can be addressed in this configuration by getting a word value and selecting
the appropriate byte by div and mod operations. Doing this, it has to be considered that MIPS is Little Endian \footnote{\verb"en.wikipedia.org/wiki/Endianness"}. \\
Numerous checks are also made in the specification of the data transfer instructions. It is checked whether the
memory access is within stack bounds. Furthermore word alignment is checked for the \verb"lw" and \verb"sw" instructions.  

\subsection{Bitwise Logical Instructions}

The bitwise logical instructions can be used to perform boolean operations or bit masking. The result of the 
bitwise operations is achieved in the Why3 model by chopping the register values into single bits by div and mod
operations. Then the corresponding boolean operation (and, or) is applied to the bits. Finally the 32-bit result
is reconstructed from the bits by multiplication and addition. 

\subsection{Conditional Branches}

Conditional branch instructions jump to a program point when a condition is met and execute the next instruction
when the condition is not met. Their Why3 equivalents return 1 if they would branch and 0 if they would not. The
value of the result is then used in a WhyML if-then-else statement that corresponds to the branch in the assembly code.

\section{Translation to WhyML}

The translation program \verb"mips2why3" maps constructs of assembly language to their counterparts in WhyML. It is 
a parser that is generated from a context free BNF grammar. \verb"mips2why3" recognizes a fusion of the MARS assembly 
syntax and an annotation language. Some annotations are simply passed through from input to output (pass through
blocks) other annotations are keywords that announce a specific language construct (for example procedures, loops
or branches). Simple as it is, \verb"mips2why3" performs no semantic checks. So some care must be taken when using it.
Nevertheless it is much faster to work with \verb"mips2why3" than performing a manual translation would be. It is 
sufficient for the purposes of training and programming experiments. \\
\verb"mips2why3" is invoked from command line with the assembly input file as argument.
\begin{lstlisting}
java -jar mips2why3.jar program.asm
\end{lstlisting} 
The program writes its output to the terminal. Writing to file is done by redirection.
\begin{lstlisting}
java -jar mips2why3.jar program.asm > program.mlw
\end{lstlisting} 
The grammar, together with a detailed description of the language, can be found on GitHub.

\section{Validation of the MIPS Model}
A set of example programs was written to validate the MIPS model (and the translation program).
For each MIPS instruction a small test was implemented. Furthermore there are tests for procedure calls,
branch constructs and loops. The tests describe simple use cases and sometimes special characteristics of 
particular instructions. The tests are MIPS assembly programs with assertions that describe the expected result
of program execution. The tests were translated to Why3 to perform the proof of all assertions. Then the tests
were executed in MARS. Finally the results were compared. All assertions could be proved (with automated provers)
\footnote{The Z3 solver could even prove assertions about the bitwise instructions, despite of the large terms involved.}
 and the results were in accordance with the execution results in MARS. 
The test cases and their description are available on GitHub.  
%%%%
