Verification of assembly programs with help of theorem provers was already done in \cite{spade}. More recent
efforts can be found in \cite{verimix} and \cite{why3-avr}.
This work has focused on an educational approach by combining MIPS assembly verification in Why3 with execution
in the MARS simulator. For this purpose a parser was implemented that helps to translate assembly in MARS syntax into 
the input language of Why3. Furthermore a theory for a subset of the MIPS instruction set was written in Why3.
The results of the tests are promising. \\
The Why3 theory helps to understand the MIPS instruction set and assembly
programming in general. The introduced static checks help to find errors early in the development process.
In this context, automated theorem provers are a great help for the programmer, since it would be impossible to perform all checks manually.\\
 As further work it is planned to adapt an approach similar to \cite{verimix} and to extract theories directly 
from the assembly code (bypassing translation to WhyML). This would give the programmer more freedom regarding the
control flow of the program.  