\section{Why3 Verification Platform}

Why3 \footnote{\verb"why3.lri.fr"} is used to talk to theorem provers. The main
application of Why3 is program verification. Why3 has an input language 
to write theories, programs and the specification for the programs. 
The input language is translated to the input languages of different theorem
provers. Then the provers are invoked to find a proof for the given goals.
Why3 can be compared to a compiler. The Why3 front end reads the input
language. Since different theorem provers have different input languages
Why3 has several back ends to translate the input to various formats.

\subsection{Why3 Workflow}

In this work the version 0.87.3 of the Why3 platform was used.
It was obtained by downloading the package \textbf{why3} with Debian Linux 9.
The GitHub repository for this project contains more information about
installation and configuration of Why3 \footnote{\verb"github.com/zul2/mips2why3"}. 
Input files for Why3 have the extension ".mlw". After writing an input file
you might want to perform a check for errors. The \verb"prove" command checks
the syntax and semantics of the input.
\begin{lstlisting}
why3 prove -L . --type-only example.mlw
\end{lstlisting}
The \verb"-L ." option adds the current working directory to the library path.
So you can use your own library files in addition to the 
standard library of Why3.
The \verb"ide" command starts an interactive proof session.
\begin{lstlisting}
why3 ide -L . example.mlw
\end{lstlisting}
In the interactive session proof goals can be selected and \emph{split} into 
smaller subgoals. \emph{Provers} can be called to solve the goals. 
The \emph{source code} of the of the input files can be viewed. Often it is also useful
to look at the \emph{task} (the theory and the goals) that was extracted from
the input and will be presented to the provers.

\subsection{Why3 Input Language}

The following part describes a subset of the Why3 input language. This
is done for two reasons. First the manual of Why3 \cite{why3man} does not always provide 
this kind of information. Second many features of the language are not
needed for the purpose of this work and are omitted. The basic concepts
are presented here and additional features are introduced when needed.
The full source code for the examples can be found at the GitHub repository.

\subsubsection{Modules}

The Why3 input language is organized in modules. A module can contain
everything the language offers (theories, programs and specification).
Entities (functions or constants for example) from other modules are included with the \verb"use" keyword. 
\begin{lstlisting}
use file_name.module_name
\end{lstlisting}
Usage of these entities requires a fully qualified name.
\begin{lstlisting}
module_name.entity_name
\end{lstlisting}
The optional \verb"import" keyword makes the entities directly visible. So full
qualification is no longer needed for the entities of the imported module.
However the possibility of name collisions raises when multiple modules are
imported.
\begin{lstlisting}
module MyModule

(* This is a Why3 comment *)

(* Why3 library *)
use import int.Int
(* User defined library*)
use a.A
use import b.B

(* x from A; y from B *)
function sum: int = A.x + y

end
\end{lstlisting}

\subsubsection{Types}

In this work the int type from the Why3 standard library
will be used. It is an unbound integer type and corresponds to the set of 
whole numbers $\mathbb{Z}$ in mathematical notation. Later an unbound array (map) of 
integers will be used to model the memory.

\subsubsection{Theories - First Order Logic}

The Why3 language for theories is first order logic. It consists of terms
and formulas \cite[p. 739]{smt}. Terms are comparable to functions in
standard programming languages. Formulas can be compared to functions with
a boolean return value (predicates). Furthermore first order logic 
supports quantifiers over elements of a set ($\forall$, $\exists$).
First order logic is described in textbooks about mathematical logic. A
good introduction for programmers is given in \cite[p. 135--150]{spark14}.

\paragraph{Terms}

A term in first order logic can be a constant, a function or a conditional
term. A constant is a function that has no parameters. 
\begin{lstlisting}
constant x: int
\end{lstlisting}
A function takes parameters and returns a value of a particular type.
A function is defined by giving it a name, defining its input parameters 
and defining its return value after a colon.
\begin{lstlisting}
function add (x: int) (y: int) (z: int): int
\end{lstlisting}
A function defined as above is considered abstract since it has no
function body. When we want to construct a function from other terms we 
have to do it in a function body.
\begin{lstlisting}
function add (x:int) (y: int) (z:int): int = x + y + z
\end{lstlisting}
A function can be called by specifying its name and its actual parameters.
The parameters can be written within parentheses.
\begin{lstlisting}
add a (b+8) 7
\end{lstlisting}
A conditional term is an if-then-else statement expecting one formula 
(predicate) and two terms as the alternatives.
\begin{lstlisting}
function abs (x:int): int = if (x >= 0) 
                            then (x) 
                            else ( (-1) * x )
\end{lstlisting}
Note that \verb"*" and \verb"+" are also functions. They are defined in the Why3 
standard library and are called using the common infix notation
("2 + 3" instead of "+ 2 3"). 

\paragraph{Formulas}

First of all a predicate is a formula in first order logic.
Predicates are comparable to functions with a boolean return value in 
common programming languages. Predicates can have terms as input arguments.
\begin{lstlisting}
predicate greater (x:int) (y:int) = x > y
\end{lstlisting}
A predicate definition is similar to a function definition. There is no
need to specify the return type since predicates (formulas) are always of
"boolean" type.
A predicate with no input arguments is comparable to a constant term.
\begin{lstlisting}
predicate today_is_a_rainy_day
\end{lstlisting}
A special formula/predicate is the equality between two terms.
\begin{lstlisting}
1 = (-1) * (-1)
\end{lstlisting}
Furthermore \verb"true" and \verb"false" are basic formulas. Formulas can be built
from other formulas by applying boolean operators.
\begin{lstlisting}
(* always true *)
a \/ true
(* always false *)
not ( a /\ false )
(* logical double negation *)
a <-> ( not ( not a ) )
(* implication rewrite *)
(a -> b) <-> ( (not a) \/ b )
\end{lstlisting}
The table shows the syntax of some common boolean operators in Why3.
Their exact meaning is defined by truth tables \footnote{\verb"en.wikipedia.org/wiki/Truth\_table"}.

\begin{tabular} {c|c|c}
Why3 & English & Name \\
\hline
\verb"a /\ b" & a and b & Conjunction \\
\verb"a \/ b" & a or b & Disjunction \\
\verb"a -> b" & from a follows b & Implication \\
\verb"a <-> b" & a is equivalent to b & Equivalence \\
\verb"not a" & not a & Negation \\
\end{tabular}

\paragraph{Quantifier}

Quantifier are used to write statements about all members of a set.
\begin{lstlisting}
forall i:int. i < (i + 1)
\end{lstlisting}
The statement above says that all integers are smaller than their 
successor. Quantifier start with the keywords \verb"forall" or \verb"exists", 
followed by a variable name and type, followed by a full stop. After
the full stop a formula is given that contains the declared variable.
Quantified statements are often used in mathematics. We can translate the
above statement into classical mathematical notation.
\[\forall i \in \mathbb{Z}: i < (i + 1)\]
Quantified statements are also formulas. So you can use them in first order
logic like predicates.
\begin{lstlisting}
(-10 < 0) /\ ( forall i:int. (i > 10) -> (i > 0) ) 
\end{lstlisting}
\verb"Forall" states that the formula is valid for all members of the given set
(Universal Quantifier: $\forall$).
\verb"Exists" states that the formula is valid for at least one member of the 
given set (Existential Quantifier: $\exists$).
Quantified statements can be nested. The following example states that
for all integers there is at least one number that is greater than this
integer.
\begin{lstlisting}
forall i:int. exists j:int. j > i
\end{lstlisting}

\paragraph{Axioms and Goals}

A theory consists of a set of definitions (predicates and functions) and 
a set of axioms. An axiom is a formula that is defined to be true.
\begin{lstlisting}
costant a: int
axiom Aval: a = 10
\end{lstlisting}
When a goal is specified a theorem prover will use the theory and try to
prove the given goal.
\begin{lstlisting}
goal A_greater_2: a > 2
\end{lstlisting}
Why3 has the same syntax for goals and axioms. Especially, you need to 
give a name to every axiom or goal.
Note that wrong or contradictory axioms break a theory. 
In this case the theory is called inconsistent.
With an inconsistent theory a prover will prove every goal regardless
whether it is valid or not.
\begin{lstlisting}
axiom Nonsense: 2 = 7
goal Nonsense2: 17 + 2 > 1000
\end{lstlisting}

\subsubsection{Programs - WhyML}

A Why3 module can also contain programs. The programming language used
by Why3 is called WhyML. Since it is a variant of ML it is a functional
programming language. Nevertheless WhyML also supports an imperative,
C like, programming style.

\paragraph{References} \label{ref}

To write imperative programs we have to use 
references from the Why3 standard library \footnote{\verb"why3.lri.fr/stdlib/"}.
\begin{lstlisting}
use import ref.Ref
\end{lstlisting}
Reference variables can be written and read. They behave like variables in
imperative languages like C. Note that a variable has to be dereferenced
with the \verb"!" operator to obtain its value.
\begin{lstlisting}
(* increment x *)
x := !x + 1
\end{lstlisting}

\paragraph{Procedures} \label{function}

WhyML functions have a similar syntax to first order logic functions. The 
\verb"function" keyword is replaced by the \verb"let" keyword.
\begin{lstlisting}
let add (x:int) (y: int) (z:int): int = x + y + z
\end{lstlisting}
We can write a procedure in WhyML by returning \verb"unit". Furthermore the parameters
of procedures and functions can be references (in out parameters).
\begin{lstlisting}
let add (x: ref int) (y: ref int) 
        (z:ref int) (r: ref int): unit
=
 r := !x + !y + !z;
 ()
\end{lstlisting}
The empty statement \verb"()" at the end of the procedure denotes return from the procedure.

\paragraph{Conditional Execution}

To express conditional execution the if-then-else statement will be used.
\begin{lstlisting}
if (!x = 10) then (
 x := !x + 1;
 y := 0
)
else (
 x := !x - 1;
 y := 1
);
\end{lstlisting}
Note that the position of the semicolons differs from C.

\paragraph{Loops}

To express loops an unconditional loop with break statements in the loop body
will be used. Break statements are realized in WhyML by exceptions.
\begin{lstlisting}
try
 loop
  if (!x = 10) then raise Break;
  x := !x + 1 
 end (* loop *)
with Break -> ()
end (* try block *) ;
\end{lstlisting}
There is no code that handles the exception. After exception raising the program simply continues 
at the point below the end of the loop.

\subsubsection{Specification - Hoare Logic}

The principle of writing the specification for a program is based on
Hoare Logic \cite[p. 65--74]{fmb}. At the core of this logic is the
concept of annotating a program with a precondition (something that is true before program execution) and a postcondition (something that is true
after program execution). Below is an abbreviation of this concept.
\begin{lstlisting}
{P} S {Q}
\end{lstlisting}
The above formula means: "When the precondition P is true
and the program S terminates the postcondition Q will be true".\\
The input of theorem provers are theories and the corresponding
goals, but not programs. So we need an algorithm that translates a program
into a theory that corresponds to the behavior of the program. 
Fortunately Why3 can do this for us. The process called Verification 
Condition Generation extracts multiple theories and goals from a WhyML
program. For this purpose Why3 analyzes the control and data flow of the program.
Here we will see how Why3 deals with the four basic cases:
\begin{itemize}
\item sequential composition of statements
\item conditional branches
\item procedure invocation
\item loops
\end{itemize}

\paragraph{Sequential Composition - Managing Program States}

Suppose we have three integer variables a, b, c and this sequence of
statements:
\begin{lstlisting}
a := 3;
a := !a + 1;
b := !a;
c := !a + !b;
\end{lstlisting}
To express proof goals in WhyML programs we can use assertions. 
Assertions are formulas of first order logic that have to be true at 
the program point where they are inserted. An assertion in a WhyML program
is translated to a goal in a Why3 theory. Assertions are expressed with the
keyword \verb"assert" followed by a formula in brackets.
\begin{lstlisting}
a := 3;
a := !a + 1;
assert { !a = 4 };
b := !a;
assert { !b = 4 };
c := !a + !b;
assert { !c = 8 };
assert { !c > 0 };
\end{lstlisting}
Using the control and data flow of the program Why3 builds a theory from
the program source. Every time it encounters an assertion it can print the
generated theory with the goal connected to the assertion. For the first
assertion we get basically the following theory:
\begin{lstlisting}
axiom H: a0 = 3
axiom H1: a1 = (a0 + 1)
goal WP: a1 = 4
\end{lstlisting}
The most important point here is that an assignment changes the state of
a variable. The old state a0 is used to compute the new state a1 of 
variable a. The assertion deals with the most recent state of the 
variable a which is a1 in this case.
Here is the theory for the second assertion:
\begin{lstlisting}
axiom H: a0 = 3
axiom H1: a1 = (a0 + 1)
axiom H2: a1 = 4
axiom H3: b = a1
goal WP: b = 4
\end{lstlisting}
Note that axiom H2 is the formula that will be proved in the first
assertion. Previous assertions are used as axioms in subsequent assertions
regardless whether they are true or not. Here Why3 assumes that you will
leave no assertion unproven. So unproven assertions can break the proof of
the rest of the program.

\paragraph{Conditional Branches}

A conditional branch splits the program in two paths. Why3 can generate a
theory for each path.
\begin{lstlisting}
if (!a < 0) then (
 a := !a * (-1);
 b := 0
)
else (
 b := 1
);

c := 7;

assert { (!a >= 0) /\ (!b = 1 \/ !b = 0) /\ (!c = 7) };
\end{lstlisting}
For the code above we get two theories. The first one deals with the case
$a < 0$.
\begin{lstlisting}
axiom H: a0 < 0
axiom H1: a1 = (a0 * (-1))
axiom H2: b0 = 0
axiom H3: c0 = 7
goal WP: a1 >= 0 /\ (b0 = 1 \/ b0 = 0) /\ c0 = 7
\end{lstlisting}
In the second theory we have $a \geq 0$.
\begin{lstlisting}
axiom H: not (a0 < 0)
axiom H1: b0 = 1
axiom H2: c0 = 7
goal WP: a1 >= 0 /\ (b0 = 1 \/ b0 = 0) /\ c0 = 7
\end{lstlisting}
The assertion has to be true for both program paths. 
Note that multiple assertions at the same position can be combined
into one assertion by making a conjunction.
So
\begin{lstlisting}
assert { A };
assert { B };
assert { C };
\end{lstlisting}
is equivalent to 
\begin{lstlisting}
assert { A /\ B /\ C };
\end{lstlisting}

\paragraph{Procedure Invocation - Preconditions and Postconditions}

Functions and procedures are the programming constructs where the concept
of Hoare Logic is explicitly visible. Here is an example:
\begin{lstlisting}
let sum (a: ref int) (b: ref int) (c: ref int): unit
 requires { (!a >= 0 /\ !a <= 10) /\ 
            (!b >= 0 /\ !b <= 10) }
 ensures { !c = old (!a) + old (!b) }
=
 c := !a + !b;
 a := 10;
 ()
\end{lstlisting}
The keywords \verb"requires" and \verb"ensures" define a special form of assertions.
\verb"Ensures" defines the precondition of a function or procedure. 
The precondition must be true at the beginning of the procedure. Every time
the procedure is invoked Why3 generates a check that the precondition
holds at this point of execution. When Why3 analyzes the body of the 
procedure the precondition is assumed to be true.
\verb"Requires" defines the postcondition of a function or procedure. The
postcondition must be true at the end of the procedure. Why3 uses the 
precondition together with the body of the procedure to generate a theory
with the goal to prove the postcondition. For our example this theory has
the following shape:
\begin{lstlisting}
axiom H : (a0 >= 0 /\ a0 <= 10) /\ (b0 >= 0 /\ b0 <= 10)
axiom H1 : c0 = (a0 + b0)
axiom H2 : a1 = 10
goal WP : c0 = (a0 + b0)
\end{lstlisting}
Only the arguments of a procedure can be used in preconditions and 
postconditions. Local variables can not be used since they are not visible
for the caller. The only exception is the return value of a function. It 
can be used in a postcondition and is called \verb"result" in Why3.
Note that our variables can change their state during execution of the 
procedure. So it is important to define which state of the variables we
mean when we use the name of a variable in preconditions and
postconditions. 
In the case of a precondition the use of a variable is unambiguous.
It means the value of a variable before execution of the procedure begins
(prestate). So in our example writing a in the precondition means 
"the value of variable a in the prestate a0".
On the contrary writing a in the postcondition means "the value of 
variable a in
its most recent state at the end of the procedure" (in our example it 
would be a1). When we want the value of a in the prestate we have to
write \verb"old (!a)" in the postcondition.
Here is an example for the invocation of our example procedure.
\begin{lstlisting}
x := 2;
y := 3;
z := 7;
sum x y z;
assert { !z = 5 };
assert { !y = 3 };
(* ! no proof ! *)
(* assert { !x = 10 \/ !x = 2}; *)
\end{lstlisting}
When \verb"sum" is called Why3 generates a theory to prove the precondition.
Therefore the formal parameters (a, b, c) are substituted by the actual
parameters (x, y, z).
\begin{lstlisting}
axiom H: x0 = 2
axiom H1: y0 = 3
axiom H2: z0 = 7
goal WP: (x0 >= 0 /\ x0 <= 10) /\ (y0 >= 0 /\ y0 <= 10)
\end{lstlisting}
When \verb"sum" returns the postcondition is assumed to be true and can be 
used as axiom to prove additional assertions.
\begin{lstlisting}
axiom H : x0 = 2
axiom H1 : y0 = 3
axiom H2 : z0 = 7
axiom Precondition : (x0 >= 0 /\ x0 <= 10) /\ 
                     (y0 >= 0 /\ y0 <= 10)
axiom Postcondition : z1 = (x0 + y0)
goal WP: z1 = 5
\end{lstlisting}
In our example the states of x and z are changed by the invocation of 
\verb"sum". On the contrary the state of y is not changed. With data flow 
analysis Why3 is able to figure this out. Why3 knows that x and z are 
changed but not how. In the case of z the postcondition provides the 
information. But since the postcondition of \verb"sum" is silent about the state
of x, the last assertion of our example is not provable. After the 
procedure invocation x is neither 2 nor 10 in our theory.
This is an illustrative example, that shows how context is handled in this analysis
method. At the beginning of a procedure we see a reduced context. The only
thing we can assume about all the possible invocations of the given
procedure is that its precondition holds. Likewise the
postcondition provides us a reduced view of the things that happen inside
a procedure. The advantage of this approach is that we can decompose
our program into smaller units that can be handled separately. Then the
preconditions and postconditions form the interface of our subunits.

\paragraph{Loops - Invariants and Variants}

\emph{Loops are a problem} \cite[p. 201]{spark14}. Loops are difficult to handle in programs
and the difficulties become even more explicit in a proof attempt.
Loop invariants and variants are the two constructs that are available
when dealing with loops. They are used to provide the rest of the program
with information about the behavior of a loop (similar to a postcondition
for a procedure). With loop variants it is possible to prove loop 
termination.

\subparagraph{Loop Invariants}

In Why3 a loop invariant is a special assertion that is placed at the
beginning of a loop. The invariant describes a predicate that is always true
during loop execution.
\begin{lstlisting}
x := 0;

try
 loop
 invariant { 0 <= !x <= 10 }
 (* begin loop *)
  if (!x = 10) then raise Break;
  x := !x + 1 
 end (* loop *)
with Break -> ()
end (* try block *) ;
assert { !x = 10 };
\end{lstlisting}
The above example describes a simple loop that increments x until it 
has the value 10. The proof of a loop invariant proceeds similar to a
proof by induction in mathematics. It consists of two steps. First we have
to prove that the invariant is valid when we first arrive at the loop. This
is called loop invariant initialization. The second step is called loop
invariant preservation. Assuming that the loop invariant is valid for
iteration n we have to show that the invariant is also valid for iteration n+1.
The process for extraction of the relevant theories can be imagined as
follows. When we analyze the program following its control flow 
(in the simplest case from top to bottom) we gather
information about the program behavior and write it down in form of
logical statements. When we enter a loop and see the loop invariant the
first time we try to prove it using the collected information. So we get
our theory for the loop invariant initialization. For our example Why3
generates the following verification condition for loop invariant
initialization.
\begin{lstlisting}
axiom H : x0 = 0
goal Loop_invariant_init : 0 <= x0 /\ x0 <= 10
\end{lstlisting}
Now we assume the loop invariant to be true. Furthermore we forget 
everything we know about the variables in the loop except what is stated
in the invariant. We don't know whether we arrived at the loop 
invariant by entering the loop the first time or by jumping back to make
another iteration. The information we gathered about the loop variables
before loop entry is not valid anymore. 
Now we write the theory for loop invariant preservation. We assume the
loop invariant and traverse the loop body until we jump back and arrive
at the same invariant again. Note that this program path is generic. It
describes many iterations of the loop not just one. In our example the following
path describes all possible iterations of the loop.
\begin{lstlisting}
axiom H1 : 0 <= x1 /\ x1 <= 10
axiom H2 : not x1 = 10
axiom H3 : x2 = (x1 + 1)
goal Loop_invariant_preservation : 0 <= x2 /\ x2 <= 10
\end{lstlisting}
Now we can prove the goals for loop invariant initialization and 
preservation. Consequently the valid loop invariant at loop entry makes the loop
invariant in the first iteration true. The loop invariant of the first
iteration makes the loop invariant of the second iteration true and so on
(induction principle). 
Finally we will see what happens when the program exits the loop. Here
again we have a generic path. We start at the loop invariant. In the 
previous proof we showed that the the loop invariant is valid for all paths
that can reach the invariant. So the invariant is true. We traverse the 
loop body and arrive at the break statement where we exit the loop.
\begin{lstlisting}
axiom H1 : 0 <= x1 /\ x1 <= 10
axiom H2 : x1 = 10
goal WP: x1 = 10
\end{lstlisting}

\subparagraph{Loop Variants}
In the previous section we did not prove loop termination. Our example
loop terminates when x becomes 10 but we did not show that x actually 
becomes 10. Our loop invariant just states that the value of x is always
between 0 and 10.
To prove loop termination we need a loop variant. In Why3 a loop variant
can be inserted like a loop invariant at the beginning of a loop.
\begin{lstlisting}
try
 loop
 invariant { 0 <= !x <= 10 }
 variant { 10 - !x }
 (* begin loop *)
  if (!x = 10) then raise Break;
  x := !x + 1 
 end (* loop *)
with Break -> ()
end (* try block *) ;
assert { !x = 10 };
\end{lstlisting}
Note that a loop variant, unlike a loop invariant, is a term in first order
logic and not a predicate. The loop variant must be non negative and 
decrease in each iteration of the loop. 
\begin{lstlisting}
axiom H1 : 0 <= x1 /\ x1 <= 10
axiom H2 : not x1 = 10
axiom H3 : x2 = (x1 + 1)
goal Loop_variant_decrease : 0 <= (10 - x1) /\ 
                             (10 - x2) < (10 - x1)
\end{lstlisting}
With a proof of the loop variant we have found a variable that has 
a lower bound (in our case 0) and decreases with every iteration of the
loop. Eventually the variant will reach value 0 and the loop will terminate.
Note that the loop invariant provides the information needed for the proof
of the variant. The loop invariant also helps to describe the state of the
program \emph{if} the program exits the loop. The proof \emph{that} the program exits the
loop is done with the help of a loop variant.
