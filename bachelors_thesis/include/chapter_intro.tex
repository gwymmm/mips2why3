\chapter{Introduction}

\section{Formal Methods}
Often programs are difficult to understand and maintain due to their
complexity and size. Nevertheless reliable programs are important for
modern technologies. 
One abundant technique to improve software quality is testing. In the
case of software, testing is relatively cheap and can be done 
extensively. However it is in general impossible to execute all possible
paths in a program. Furthermore only an already implemented program can
be tested. Thus problems and errors are detected late in the development
leading to major rework and waste of resources. Additional techniques
are needed to perform software development in a satisfactory way.
Since every program is a calculation, mathematics can be used throughout
the development to improve software quality. Techniques that use 
mathematics for software development are called formal methods \cite{fmb}.
Formal methods have in common that they provide means to calculate the
behavior of a program \cite[p. 371]{scs}. This stands in contrast to 
testing where the program is executed and its behavior observed. With
formal methods it is possible to prove that statements about program
behavior are true for all possible states, whereas testing observes it
for single cases.
One particular formal method is based on programming by contract 
\footnote{\verb"en.wikipedia.org/wiki/Design\_by\_contract"}. 
First a program is annotated with contracts. In a second step the 
contracts are proven to hold for this program. For some languages there
is already tool support for this technique. Especially the recent 
advances in automated theorem proving make the use of theorem provers
attractive in the context of program verification \cite{onelogic}.
A thorough analysis is important when
\begin{itemize}
\item the calculation is large and repeating it is painful
\item important decisions depend on the result of the calculation
\item the program is part of a system that must operate autonomously 
at remote locations
\item system failure leads to huge losses
\end{itemize}
Formal methods can assist the engineers and scientists who perform
the calculations and build the systems.

\section{Assembly Verification}

Often only the source code of a high-level programming language is 
analyzed when writing a program. Nevertheless it can also be beneficial
to extend the (formal) verification to the assembly or object code level.
Since compiler are large complex programs their implementation needs a
lot of effort. For some architectures no compiler might be available. So
you have to program in assembly. Hand coded assembly procedures might
also be necessary to speed up a calculation or to implement a device
driver \cite[p. 4 -- 8]{armb}. A method that helps in the verification of an assembly
program is highly welcome in these cases. Moreover the analysis of assembly
or object code can relieve engineers of the burden to prove correctness
of a compiler.
Finally the behavior of a machine with a von Neumann architecture is 
relatively simple to understand. Especially when the instruction set is
kept simple, like in the case of the MIPS architecture. High-level 
languages on the contrary have many constructs and rules that are hard
to understand and to learn. Sometimes programming languages have no clear
semantics at all (e. g. Undefined Behavior in C 
\footnote{\verb"blog.regehr.org/archives/213"}). Thus assembly
is more amenable to formal analysis.

\section{Why3}

Why3 enables verification of programs with help of theorem provers \cite{wpmp}.
It will be used in this work to analyze MIPS assembly programs. The
verification method used in Why3 is a variant of design by contract.
Assertions are used to describe what a program should do (specification).
For this purpose a specification language is used that allows the 
description of the expected program behavior using logical statements
(e. g. "All elements of array A have a value between 1 and 10.").
Furthermore a programming language (WhyML) is used to define what the
program actually does (implementation). An algorithm translates the program into mathematical
formulas. For this purpose it analyzes the control and data flow of the program. As an example the 
sequence of assignments $x := 3; x := x + 1;$ could be translated to 
the logical statements $x_1 = 3 \text{ and } x_2 = x_1 + 1$. The user defined 
assertions also represent mathematical formulas. So Why3 can invoke a
theorem prover in order to prove that these assertions can be proved
from the formulas corresponding to the implementation of the program. In
the above example the annotated program
\begin{lstlisting}
    x := 3;
    x := x + 1;
    assert x = 4
\end{lstlisting}
would be translated to 
\begin{lstlisting}
    proof context: x_1 = 3 and x_2 = x_1 + 1
    proof goal: x_2 = 4
\end{lstlisting}
The proof can succeed by substitution of $x_1 = 3$ in $x_2 = x_1 + 1$.
So we get $x_2 = 3 + 1 = 4$.
A successful proof is a confirmation that the implementation of the 
program is in accordance with its specification.


